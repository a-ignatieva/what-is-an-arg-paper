% LaTeX rebuttal letter example.
%
% Copyright 2019 Friedemann Zenke, fzenke.net
%
% Based on examples by Dirk Eddelbuettel, Fran and others from
% https://tex.stackexchange.com/questions/2317/latex-style-or-macro-for-detailed-response-to-referee-report
%
% Licensed under cc by-sa 3.0 with attribution required.

\documentclass[11pt]{article}
\usepackage[utf8]{inputenc}
\usepackage{lipsum} % to generate some filler text
\usepackage{fullpage}

% import Eq and Section references from the main manuscript where needed
% \usepackage{xr}
% \externaldocument{manuscript}

% package needed for optional arguments
\usepackage{xifthen}
% define counters for reviewers and their points
\newcounter{reviewer}
\setcounter{reviewer}{0}
\newcounter{point}[reviewer]
\setcounter{point}{0}

% This refines the format of how the reviewer/point reference will appear.
\renewcommand{\thepoint}{\thereviewer.\arabic{point}}

% command declarations for reviewer points and our responses
\newcommand{\reviewersection}{\stepcounter{reviewer} \bigskip \hrule
                  \section*{Reviewer \thereviewer}}

\newenvironment{point}
   {\refstepcounter{point} \bigskip \noindent {\textbf{Reviewer~Point~\thepoint} } ---\ }
   {\par }

\newcommand{\shortpoint}[1]{\refstepcounter{point}  \bigskip \noindent
    {\textbf{Reviewer~Point~\thepoint} } ---~#1\par }

\newenvironment{reply}
   {\medskip \noindent \begin{sf}\textbf{Reply}:\  }
   {\medskip \end{sf}}

\newcommand{\shortreply}[2][]{\medskip \noindent \begin{sf}\textbf{Reply}:\  #2
    \ifthenelse{\equal{#1}{}}{}{ \hfill \footnotesize (#1)}%
    \medskip \end{sf}}

\begin{document}

\section*{Response to the editor}
% General intro text goes here
Thank you for considering this manuscript for publication. 

% Associate Editor's overview & review
\reviewersection

\begin{point}
I gained a number of insights from reading the paper, however, I wonder who the
intended audience of the paper is. It is not a research paper, indeed many of
the technical details are given in other papers. Nor is it written at a
particularly accessible level for people trying to understand ARGs and tree
sequences.The paper is advocating for the adoption of "genome ARG (gARG)" and
its simplified versions, but the authors need to be much clearer who they want
to adopt this and in what level. Separating our structure for storing large
ARGs estimated from data, from the ``true" ARG seems like a very useful argument
to make but this gets lost a lot in the details.
\end{point}
\begin{reply}
We have clarified the purpose of the article by explicitly identifying 
``ARG practitioners'' as the intended audience. We hope that it may
serve to clarify some points of confusion among this community,
and provide a useful starting point into a large and confusing literature
for those new to the field.
\end{reply}

\begin{point}
The ``event ARG'' is hard for people to grapple with on first brush, however, in
my view, the authors' presentation here risks making ARGs even less intuitive.
\end{point}
\begin{reply}
We have attempted to simplify this by cutting out extraneous detail and 
focusing on the most important points that distinguish an Event ARG
from a Genome ARG.
\end{reply}

\begin{point}
My view is that the authors should switch the article to being a "Perspective"
or "Review" and shorten it by moving a lot of the technical detail to the
appendices. I suspect this switch will help people understand that this is the
author's take on a topic, and also help it be read through by a far broader
audience. People interested in the details can then work through the appendices
without getting lost in them on the first read through. I would be interested
to see such an article published in Genetics, as I think it would be an
important contribution and help clarify a complex field. I'm happy for the
authors to argue for keeping different sections in the main paper, but in doing
so they need to be more clearly justified to the reader in the paper.
\end{point}
\begin{reply}
We have followed this approach, and kept only what we regard as sections
essential to the main messages of the paper in the main text. Briefly,
we clearly need the Genome ARGs and Event ARGs sections, and the
Ancestral material and sample resolution section is critical to understanding 
how they relate to each other. We have added a new ``A Diversity of
Structures'' section, which highlights some important nuances possible
when using gARGs as illustrated by Figures 4 and 5. We regard Section 6 as 
vital, as it explicitly draws the connection between gARGs, eARGs and the 
tskit library.
\end{reply}

\begin{point}
The authors make a strong case that the data does not contain information about
certain aspects of the ARG, that does not mean that probabilistic statements
about these events cannot be made under particular models (or that these models
cannot be learnt from data). Coalescent inference often involves integrating
out events that we cannot observe in the data and dealing with great
uncertainty in the timing of events. I think the authors should perhaps be more
open to the view that for some applications we may want to build back in more
events even if they are not "needed" in some minimum construction. At the
moment it reads a little dogmatically, when I suspect that is not the authors
true position.
\end{point}
\begin{reply}
We have cut this text.
\end{reply}

\begin{point}
It seems in general that many in the field, at least methods-savvy folks, are
on board with tree sequences representation of data. If the "succinct tree
sequence data structure (usually known as a "tree sequence" for brevity) is a
practical gARG implementation", why not just leave "tree sequence" as the term
we use. It is far less clear to me that we need to label yet another object an
ARG, without muddying the definition of an ARG even further. More justification
is needed to help the reader understand what the authors are arguing here.
\end{point}
\begin{reply}
This was our initial position, but it has become clear that ``tree sequence''
is regularly confused with ``a sequence of disconnected trees'' and there 
are numerous inaccurate statements along these lines in the literature. 
There was a deep-seated
confusion about the relationship between what is classically considered an 
ARG, and the structures that Relate, tsinfer and ARG-Needle produce. We feel
strongly that the only way to resolve this is to show that ``tree sequences''
\emph{are} ARGs, and that there needs to be \emph{some} terminology to distinguish
the classical ``full'' ARG that is driven by events, and the more 
general structure used by tskit. Thus, in order to tackle this 
``tree sequences aren't proper ARGs'' sentiment, the terminology must contain
``ARG''. Our hope is that the term Genome ARG will not be necessary
in the future, and we will simply say ``ARG''.

We hope that the shorter main text will make the requirement for a 
simple general definition of an ARG data structure clear, without 
needing to explicitly mention these concerns in the text.
\end{reply}

\begin{point}
 The authors could also do with having a clearer separation of the "gARG" from the
"fully simplified gARG" as for a lot of the paper it feels like they informally
move back and forth of arguing the strengths of both without being fully clear
on which they are talking about.
\end{point}
\begin{reply}
We have clarified this point in several places by stating when we assume a 
fully simplified ARG and not.
\end{reply}

\begin{point}
Introduction

The authors state that concepts : "that we must be careful to distinguish are
the "true" ARG, describing the actual history of a sample of genomes, and a
"population" ARG which is the
true ARG of every individual in a population.". But given that they're both
"true" ARGs, one for a sample and one for the whole population sample it's not
clear what is that different here. Also from the description I think the
authors mean the population ARG to not include every individual who has ever
lived, but it's not clear.

"population-scale true ARGs unquestionably exist, they can also never be
entirely known," that's true of all sample ARGs

"perfectly resolved into binary splits and mergers" - not clear which direction
time is running in this statement.
\end{point}
\begin{reply}
We have cut these passages.
\end{reply}

\begin{point}
"For example, gene conversion could be accommodated either by stipulating a
third type of event (annotated by two breakpoints and corresponding traversal
conventions for recovering the local trees)" -Cite the coalescent with gene
conversion papers
\end{point}
\begin{reply}
Thanks for pointing out this omission.
We have cited Wiuf and Hein, 2000, but are not aware of other coalescent 
with gene conversion papers. We would be happy to cite others if you 
provide references.
\end{reply}

\begin{point}
"we will refer to this Griffiths encoding as the "event ARG" - It feels a
little strange to attach a single person's name to an approach that the paper
takes a fairly negative take on. I understand there's no ill feeling here, but
it's not clear how well this comes across.
\end{point}
\begin{reply}
Thanks for pointing this out --- you are absolutely right. We only mention
Griffiths in a few places now, and avoid attaching his name to the structure.
\end{reply}

\begin{point}
Briefly define "Big ARG" and "Little ARG" in the main text as they are referred
to multiple times.
\end{point}
\begin{reply}
They are no longer mentioned in the main text.
\end{reply}

\begin{point}
Section 4 and Figure 3. - There's a lot of detail in this section to then get
to this sentence at the end: "Note that the Wiuf and Hein (1999b) eARG in Fig.
3 is not particularly representative, because inference or simulation methods
usually only generate ARGs containing nodes and edges ancestral to the sample".
So this feels like the authors have attached a lot of baggage to the eARG that
it didn't actually have.
\end{point}
\begin{reply}
We have cut this paragraph. 
\end{reply}

\begin{point}
Sections 5 \& 6 similarly are fairly technical and it's not fully clear who they
are aimed at, Could these sections be moved to the appendix?
\end{point}
\begin{reply}
This has been done.
\end{reply}

\begin{point}
Section 7 and 8: Again there is a lot of detail here, much of which amounts to
a long description of Figure 4. 
\end{point}
\begin{reply}
We have moved these to appendices.
\end{reply}

\begin{point}
It is not fully clear if the authors want the
bottom graph in Figure 4 to be the end product (a "fully simplified gARG"), if
that is the case they need to state this up front and label it that way in the
graph.
\end{point}
\begin{reply}
We have added a clarification to the caption to say that this is `often
described as ``fully simplified"'.
\end{reply}

\begin{point}
 As I mention above, it's not really clear to me that labeling 4D an ARG
is really useful, given how much has been stripped away.
\end{point}
\begin{reply}
On this point we disagree. As discussed above and argued in the text, 
we feel that it is imperative
that there is a simple formal definition of an ARG that corresponds to
informal usage. Having some arbitrary threshold of ``sufficient structure''
for something to be considered a `proper' ARG can surely only lead to more
confusion. The only reasonable path forward is for the term ARG to become
general, and for the community to invent adjectives to classify
the different types of ARG.
\end{reply}

\begin{point}
Section 9: I'm not sure what the goal of this section is.
\end{point}
\begin{reply}
This section has been moved to an appendix. It is intended to clarify
the common features of existing ARG inference methods, as well as 
highlight their differences.
\end{reply}

\begin{point}
Discussion:
"Fully decoupling the general concept of an ARG from the coalescent with
recombination (hence-forth, "coalescent") is an important step." Not sure it
matters whether the coalescent
is a good or bad model for data. What matters is the separation of the idea of
genetic genealogy from stochastic models that generate such structures. That
separation occurred long before this paper, and I'm not sure that discussing
the limitations of the stochastic model adds much to what is already a long
paper. More generally, please work through the discussion with the aim of
significantly shortening it.
\end{point}
\begin{reply}
We have cut this passage and several others, substantially shortening 
the discussion.
\end{reply}


\subsection*{Specific comments}
% We address your specific comments below:
% % \reviewersection

\section*{Response to the reviewers}
% General intro text goes here
We thank the reviewers for their close reading of our manuscript and
insightful comments. In the following we address the points raised
in turn.

% Let's start point-by-point with Reviewer 1
\reviewersection

\begin{point}
In this article, Wong et al. describe a data structure called the ``genome
ARG''
(gARG), designed to effectively represent information in ancestral
recombination graphs. The authors have done an excellent job of clearly
explaining the concept and offering a detailed comparison of different ARG
definitions. The practicality of the gARG encoding is already well recognized
in the community through various applications, such as msprime, forward-time
simulation, and the efficient computation of population genetic statistics.
\end{point}
\begin{reply}
Thank you for the kind words, we are glad you find the exposition clear.
\end{reply}

\begin{point}
However, I have reservations about categorizing this work as a standalone
``Investigation'' article. It seems to align more closely with the nature of a
review article. While the authors claim to be "proposing an alternative
mathematical formulation," this characterization might be somewhat of an
overstatement. The contribution, while valuable, appears to be more in the
realm of summarizing and clarifying existing concepts rather than introducing a
fundamentally new mathematical approach.
\end{point}
\begin{reply}
We are absolutely in agreement, and have restructured the article as a
``Perspective''.
\end{reply}

\begin{point}
The figures provided are generally helpful; however, the font size in some of
them is quite small, making them difficult to read. For instance, the node
labels in Figure 3C are impossible to read without zooming in maximally. I
suggest increasing the font size in most figures to ensure clear legibility.
\end{point}
\begin{reply}
We have modified Figure 3 to make it more legible, and will endeavour to 
adapt the figures to be legible in the final published typesetting 
(should it be accepted!).
\end{reply}


\end{document}
