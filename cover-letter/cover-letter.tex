\documentclass{letter}

\signature{Jerome Kelleher}

\address{Big Data Institute\\University of Oxford}
\begin{document}

\begin{letter}{GENETICS}

\opening{Dear Editors,}

I am writing on behalf of my coauthors to submit our manuscript entitled
\emph{A general and efficient representation of ancestral recombination graphs}
for consideration for publication as a
full-length article in the
``Theoretical Population and Evolutionary Genetics'' section.

Ancestral recombination graphs (ARGs) are currently the subject of
intense research interest, with new inference methods, evaluations,
and applications regularly appearing.
In their recent preprint\footnote{The era of the ARG:
an empiricist’s guide to ancestral recombination graphs, \emph{arXiv}, 2023}
 (a biologically oriented primer
on ARGs), Lewanski et al.\ have hailed the ``Era of the ARG'', and
argue eloquently for their widespread application.

Despite this general optimism, however, we believe there is a
a fundamental problem in the area, which will substantially hamper
progress if not addressed.
The crux of the matter is that it is no longer clear
what an ARG \emph{is}, precisely.
While the classical Griffiths definition that
an ARG is the set of coalescence and recombination events
in the history of a sample is straightforward,
it is not sufficient to describe the output of methods
such as \texttt{msprime}, \texttt{tsinfer}, \texttt{Relate}
or \texttt{ARGneedle} (all of which are now routinely
referred to as ``ARG'' methods). This mismatch
between formal definition, informal usage and the
actual output of current methods is the source of deep
and persistent confusion.

We show that the classical event-based approach is
fundamentally limited, and propose a simple new
formal framework which we call the ``gARG''
encoding. We demonstrate that it generalises
the classical definitions, is fully compatible
with the output of the methods listed above,
% clarifies inference goals,
and opens up many avenues for future research.

As this this manuscript addresses a question
that is central to current research trends,
we believe that it will be of substantial interest
to the readers of GENETICS. We hope that you agree.


\closing{Sincerely,}

\end{letter}
\end{document}
